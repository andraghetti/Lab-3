\chapter*{Lab - 03}

\section{Parte 1 - Blender}
Per questa esercitazione (e per le future) si è scelto di creare una batteria. Le tecniche utilizzate per questa creazione sono state le seguenti:
\begin{itemize}
  \item Extrusion
  \item Spinning
  \item Skinning?
  \item Swinging?
\end{itemize}
Elementi della batteria creati:
  \begin{multicols}{3}
    \begin{itemize}
      \item Grancassa con aste
	  \item Hit-Hat con pedale
	  \item Rack con tutti i sostegni
	  \item Rullante
	  \item Timpano
	  \item Due Tom
	  \item Due Crash
	  \item Ride
	  \item Pedale Grancassa
    \end{itemize}
  \end{multicols}
Il risultato è il seguente. Nella cartella \textit{PART-I/Immagini del processo/} è possibile osservare vari step della creazione.
\begin{figure}[hbt]
    \centering
    \subfloat[Fronte]{{\includegraphics[height=6cm]{final1} }}%
    \subfloat[Retro]{{\includegraphics[height=6cm]{final2} }}%
	\vspace{-0.2cm}
\end{figure}

\subsection{Extrusion}
L'estrusione è stata utilizzata praticamente per tutte le parti della batteria esclusi i piatti. Selezionando una faccia o i vertici che la compongono è possibile estruderla nella direzione della sua normale, permettendo di costruire in modo incrementale delle mesh. Un esempio specifico si ha nella creazione delle aste che fermano la grancassa e nel rack:
\begin{figure}[hbt]
    \centering
    \subfloat[Durante l'estrusione]{{\includegraphics[height=6cm]{asta1} }}%
    \subfloat[Asta ultimata]{{\includegraphics[height=6cm]{asta2} }}%
	\vspace{-0.2cm}
\end{figure}

\begin{figure}[hbt]
    \centering
    \subfloat[Durante l'estrusione]{{\includegraphics[height=6cm]{rack1} }}%
    \subfloat[Rack ultimato]{{\includegraphics[height=6cm]{rack2} }}%
	\vspace{-0.2cm}
\end{figure}

Per molti elementi creati, si è utilizzata la funzione di mirroring per avere simmetria negli oggetti che lo richiedevano. Si può notare nell'immagine del Rack: Blender permette di creare e modificare una parte e la sua speculare viene creata automaticamente.


\newpage

\subsection{Spinning}
Per la parte di \textit{spinning} si è scelto di creare un piatto della batteria. Partendo da una curva NURBS è possibile modellare il profilo del piatto. Attivando la visualizzazione ortografica e trascinando sul piano di lavoro un'immagine del piatto, è possibile avere una base su cui modellare il piatto. (figura \ref{fig:nurbs})

 \begin{figure}[htb]
    \centering
    %\vspace{-0.7cm}
    \includegraphics[width=\textwidth]{nurbs}
    \caption{\label{fig:nurbs}}
    %\vspace{-0.3cm}
\end{figure}

Una volta ottenuto il profilo perfetto, occorre convertire la curva in un oggetto mesh (alt + C) per poter effettuare lo \textit{spinning}. 

\begin{wrapfigure}{l}{0.3\textwidth} %this figure will be at the rightù
    \centering
    \vspace{-0.7cm}
    \includegraphics[height=5.5cm]{spin-values}
    \caption{\label{fig:spin-values}}
    \vspace{-3.7cm}
\end{wrapfigure}

\vspace{1cm}\noindent Come si può notare in figura \ref{fig:spin-values} è stata applicata una rotazione di 360\degree, è stato preso un centro di rotazione leggermente rialzato con un asse di rotazione unico $Z$.
La curva è stata traslata leggermente verso sinistra per poter lasciare un buco in alto. Questo buco serve per inserire il piatto su un asta che lo sorregge.\\

\vspace{3cm}Il risultato ottenuto è il seguente. Si può notare la differenza tra la vesione flat e la versione smooth. Nell'immagine con la versione flat si può notare ancora la NURBS trasformata in mesh.\\

\begin{figure}[hbt]%
	\vspace{-1cm}
    \centering
    \subfloat[Piatto Flat]{{\includegraphics[height=3.5cm]{cymbal-flat} }}%
    \subfloat[Piatto Smooth]{{\includegraphics[height=3.5cm]{cymbal-smooth} }}%
	\vspace{-0.2cm}
\end{figure}


\newpage
%===========================
\section{Parte 2 - MeshLab}
